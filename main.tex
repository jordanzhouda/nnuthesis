% 使用 xelatex 编译
%\documentclass[UTF8,a4paper,zihao=-4]{ctexrep}
\documentclass[UTF8,a4paper,twoside,zihao=-4]{ctexrep}
%%%% 文档类设置 %%%%%%%%
\ctexset{
	contentsname = {目\quad 录},
	chapter = {
		beforeskip = 0pt,
		name = {第,章},
		nameformat = \zihao{3}\bfseries,
		number = \Chinese{chapter},
		numberformat = \zihao{3}\bfseries,
		titleformat = \zihao{3}\bfseries,
		fixskip = true
		        },
	section = {
		format = \raggedright,
		nameformat = \zihao{-3}\bfseries,
		numberformat = \zihao{-3}\bfseries,
		titleformat = \zihao{-3}\bfseries
			}
	    }

%%%% 宏包 %%%%%%%%%%%
\usepackage{amsmath,amssymb,amsfonts}
\usepackage{graphicx}
\usepackage{ulem}
\usepackage[colorlinks,linkcolor=blue,bookmarks=true]{hyperref}
\usepackage{pdfpages}
\usepackage{lipsum}
\usepackage[bindingoffset=.5cm,centering,includeheadfoot,margin=2.5cm]{geometry}
\usepackage{fancyhdr}
\pagestyle{fancy}
\renewcommand{\headrulewidth}{0pt}

%%%% 字体 %%%%%%%%%
\usepackage{lmodern}
\setCJKmainfont[
	BoldFont=AdobeHeitiStd-Regular.otf,
	ItalicFont=AdobeKaitiStd-Regular.otf
				]{simsun.ttc}
%%%%%%%%%%%%%%%%%%%%%%
%%%%%%%%%%%%%%%%%%%%%%
%%%%%%%%%%%%%%%%%%%%%%
\begin{document}
\includepdf{cover} %%%%% 封面 %%%%%%%%%%%%
%%%%%%% 中英文摘要 %%%%%%%%%%%%%%%%
%%% 中文摘要
\clearpage
\thispagestyle{plain}\pagenumbering{roman}
\phantomsection
\addcontentsline{toc}{chapter}{摘\quad 要}

\centerline{\zihao{-3}\heiti 摘\quad 要}

\linespread{1.4}\zihao{-4} \bigskip

本文给出一个简易的南师大本科毕业论文\LaTeX 排版示例。

没有版权。

\bigskip

\noindent{\zihao{4}\heiti 关键词:}
\LaTeX, 排版

% Abstract
\clearpage
\thispagestyle{plain}
\phantomsection
\addcontentsline{toc}{chapter}{Abstract}

\centerline{\zihao{3}\bfseries Abstract}

\linespread{1.4}\zihao{-4}
\bigskip

This article provides a simple showcase of tysetting bachelor thesis of Nanjing Normal University with \LaTeX.

No rights reserved.

\bigskip
\noindent\textbf{\zihao{4} Keywords:} 
\LaTeX, Typesetting


\tableofcontents\newpage\mbox{}\thispagestyle{empty}\newpage
\clearpage\pagenumbering{arabic}
\chapter{中文排版}
来一回《三国演义》
\section{宴桃园豪杰三结义\quad 斩黄巾英雄首立功}
话说天下大势,分久必合,合久必分。周末七国分争,并入于秦。及秦灭之后,楚、汉分争,又并入于汉。汉朝自高祖斩白蛇而起义,一统天下,后来光武中兴,传至献帝,遂分为三国。推其致乱之由,殆始于桓、灵二帝。桓帝禁锢善类,崇信宦官。及
桓帝崩,灵帝即位,大将军窦武、太傅陈蕃共相辅佐。时有宦官曹节等弄权,窦武、陈蕃谋诛之,机事不密,反为所害,中涓自此愈横\footnote{这是一个脚注。}。

建宁二年四月望日,帝御温德殿。方升座,殿角狂风骤起。只见一条大青蛇,从梁上飞将下来,蟠于椅上。帝惊倒,左右急救入宫,百官俱奔避。须臾,蛇不见了。忽然大雷大雨,加以冰雹,落到半夜方止,坏却房屋无数。建宁四年二月,洛阳地震;又海水泛溢,沿海居民,尽被大浪卷入海中。光和元年,雌鸡化雄。六月朔,黑气十余丈,飞入温德殿中。秋七月,有虹现于玉堂;五原山岸,尽皆崩裂。种种不祥,非止一端。帝下诏问群臣以灾异之由,议郎蔡邕上疏,以为蜺堕鸡化,乃妇寺干政之所致,言颇切直。帝览奏叹息,因起更衣。曹节在后窃视,悉宣告左右;遂以他事陷邕于罪,放归田里。后张让、赵忠、封谞、段珪、曹节、侯览、蹇硕、程旷、夏恽、郭胜十人朋比为奸,号为``十常侍''。帝尊信张让,呼为``阿父''。朝政日非,以致天下人心思乱,盗贼蜂起。
\subsection{加一级标题}

时巨鹿郡有兄弟三人,一名张角,一名张宝,一名张梁。那张角本是个不第秀才,因入山采药,遇一老人,碧眼童颜,手执藜杖,唤角至一洞中,以天书三卷授之,曰:``此名《太平要术》,汝得之,当代天宣化,普救世人;若萌异心,必获恶报。''角拜问姓名。老人曰:``吾乃南华老仙也。''言讫,化阵清风而去。角得此书,晓夜攻习,能呼风唤雨,号为``太平道人"。中平元年正月内,疫气流行,张角散施符水,为人治病,自称``大贤良师"。角有徒弟五百余人,云游四方,皆能书符念咒。次后徒众日多,角乃立三十六方,大方万余人,小方六七千,各立渠帅,称为将军;讹言:``苍天已死,黄天当立;岁在甲子,天下大吉。"令人各以白土书``甲子"二字于家中大门上。青、幽、徐、冀、荆、扬、兖、豫八州之人,家家侍奉大贤良师张角名字。角遣其党马元义,暗赍金帛,结交中涓封谞,以为内应。角与二弟商议曰:``至难得者,民心也。今民心已顺,若不乘势取天下,诚为可惜。"遂一面私造黄旗,约期举事;一面使弟子唐周,驰书报封谞。唐周乃径赴省中告变。帝召大将军何进调兵擒马元义,斩之;次收封谞等一干人下狱。张角闻知事露,星夜举兵,自称``天公将军",张宝称``地公将军",张梁称``人公将军"。申言于众曰:``今汉运将终,大圣人出。汝等皆宜顺天从正,以乐太平。''四方百姓,裹黄巾从张角反者四五十万。贼势浩大,官军望风而靡。何进奏帝火速降诏,令各处备御,讨贼立功。一面遣中郎将卢植、皇甫嵩、朱儁,各引精兵、分三路讨之。

且说张角一军,前犯幽州界分。幽州太守刘焉,乃江夏竟陵人氏,汉鲁恭王之后也。当时闻得贼兵将至,召校尉邹靖计议。靖曰:``贼兵众,我兵寡,明公宜作速招军应敌。''刘焉然其说,随即出榜招募义兵。

榜文行到涿县,引出涿县中一个英雄。那人不甚好读书;性宽和,寡言语,喜怒不形于色;素有大志,专好结交天下豪杰;生得身长七尺五寸,两耳垂肩,双手过膝,目能自顾其耳,面如冠玉,唇若涂脂;中山靖王刘胜之后,汉景帝阁下玄孙,姓刘名备,字玄德。昔刘胜之子刘贞,汉武时封涿鹿亭侯,后坐酎金失侯,因此遗这一枝在涿县。玄德祖刘雄,父刘弘。弘曾举孝廉,亦尝作吏,早丧。玄德幼孤,事母至孝;家贫,贩屦织席为业。家住本县楼桑村。其家之东南,有一大桑树,高五丈余,遥望之,童童如车盖。相者云:"此家必出贵人。"玄德幼时,与乡中小儿戏于树下,曰:``我为天子,当乘此车盖。"叔父刘元起奇其言,曰:``此儿非常人也!"因见玄德家贫,常资给之。年十五岁,母使游学,尝师事郑玄、卢植,与公孙瓒等为友。

及刘焉发榜招军时,玄德年已二十八岁矣。当日见了榜文,慨然长叹。随后一人厉声言曰:``大丈夫不与国家出力,何故长叹?"玄德回视其人,身长八尺,豹头环眼,燕颔虎须,声若巨雷,势如奔马。玄德见他形貌异常,问其姓名。其人曰:``某姓张名飞,字翼德。世居涿郡,颇有庄田,卖酒屠猪,专好结交天下豪杰。恰才见公看榜而叹,故此相问。"玄德曰:``我本汉室宗亲,姓刘,名备。今闻黄巾倡乱,有志欲破贼安民,恨力不能,故长叹耳。"飞曰:``吾颇有资财,当招募乡勇,与公同举大事,如何。"玄德甚喜,遂与同入村店中饮酒。

正饮间,见一大汉,推着一辆车子,到店门首歇了,入店坐下,便唤酒保:``快斟酒来吃,我待赶入城去投军。"玄德看其人:身长九尺,髯长二尺;面如重枣,唇若涂脂;丹凤眼,卧蚕眉,相貌堂堂,威风凛凛。玄德就邀他同坐,叩其姓名。其人曰:``吾姓关名羽,字长生,后改云长,河东解良人也。因本处势豪倚势凌人,被吾杀了,逃难江湖,五六年矣。今闻此处招军破贼,特来应募。"玄德遂以己志告之,云长大喜。同到张飞庄上,共议大事。飞曰:``吾庄后有一桃园,花开正盛;明日当于园中祭告天地,我三人结为兄弟,协力同心,然后可图大事。"玄德、云长齐声应曰:``如此甚好。"

次日,于桃园中,备下乌牛白马祭礼等项,三人焚香再拜而说誓曰:``念刘备、关羽、张飞,虽然异姓,既结为兄弟,则同心协力,救困扶危;上报国家,下安黎庶。不求同年同月同日生,只愿同年同月同日死。皇天后土,实鉴此心,背义忘恩,天人共戮!"誓毕,拜玄德为兄,关羽次之,张飞为弟。祭罢天地,复宰牛设酒,聚乡中勇士,得三百余人,就桃园中痛饮一醉。来日收拾军器,但恨无马匹可乘。正思虑间,人报有两个客人,引一伙伴当,赶一群马,投庄上来。玄德曰:``此天佑我也!"三人出庄迎接。原来二客乃中山大商:一名张世平,一名苏双,每年往北贩马,近因寇发而回。玄德请二人到庄,置酒管待,诉说欲讨贼安民之意。二客大喜,愿将良马五十匹相送;又赠金银五百两,镔铁一千斤,以资器用。

\chapter{Typing English}
Just some nonsense\dots
\section{Beautiful Fonts}
\lipsum
\chapter{\LaTeX\ \&\ Mathematics}
\section{Euler Constant}
\[
\begin{aligned}
	\gamma&=\lim_{n\to\infty}\left(-\ln n+\sum_{k=1}^{n}\frac{1}{k}\right)\\
		   &=\int^{\infty}_{1}\left(\frac{1}{\lfloor x\rfloor}-\frac{1}{x}\right) dx
\end{aligned}
\]
\chapter{Conclusion}
\end{document}