% 使用 xelatex 编译
\documentclass[UTF8,a4paper,twoside,zihao=-4]{ctexrep}
%%%% 文档类设置 %%%%%%%%
\ctexset{
	contentsname = {目\quad 录},
	chapter = {
		beforeskip = 0pt,
		afterskip = 20pt,
		name = {第,章},
		nameformat = \zihao{3}\bfseries,
		number = \Chinese{chapter},
		numberformat = \zihao{3}\bfseries,
		titleformat = \zihao{3}\bfseries,
		fixskip = true,
		%afterindent = false
		        },
	section = {
		beforeskip = \ccwd,
		afterskip =1ex plus 0.2ex,
		format = \raggedright,
		nameformat = \zihao{4}\bfseries,
		numberformat = \zihao{4}\bfseries,
		titleformat = \zihao{4}\bfseries,
		%afterindent = false
			},
	subsection = {
		beforeskip =\ccwd,
		afterskip = 0.5ex,
		format = \raggedright,
		nameformat = \zihao{-4}\bfseries,
		numberformat = \zihao{-4}\bfseries,
		titleformat = \zihao{-4}\bfseries
			}
	    }

%%%% 宏包 %%%%%%%%%%%
\usepackage{amsmath,amssymb,amsfonts,mathrsfs}
\usepackage{graphicx}
\usepackage{tikz}
\usepackage{ulem}
\usepackage{pdfpages}
\usepackage{lipsum}
\usepackage{booktabs}
\usepackage{enumerate }
\usepackage[colorlinks,linkcolor=blue,citecolor=blue,bookmarks=true,bookmarksnumbered=true]{hyperref}
\usepackage[round]{natbib}
%%%%% 定理环境 %%%%%%%%%
\usepackage[thmmarks]{ntheorem}
{
  \theoremstyle{nonumberplain}
  \theoremheaderfont{\indent\bfseries}
  \theorembodyfont{\normalfont}
  \theoremsymbol{\ensuremath{\Box}}
  \newtheorem{proof}{证明}
}
{
  \theoremheaderfont{\indent\bfseries}
  \theorembodyfont{\normalfont}
  \newtheorem{theorem}{定理}[chapter]
  \newtheorem{lemma}{引理}[chapter]
  \newtheorem{definition}{定义}[chapter]
}
%%%%%% 图表标题设置 %%%%%%%%%%%%%%%
\usepackage{caption}
\DeclareCaptionLabelFormat{mylabel}{{\xeCJKsetup{CJKecglue={\hskip 0pt}}#1#2}}
\captionsetup{
		font= small,
		labelfont = bf,
		labelsep = space,
		labelformat = mylabel
			}
\makeatletter
\renewcommand{\thefigure}{\ifnum \c@chapter>\z@ \thechapter-\fi \@arabic\c@figure}
\renewcommand{\thetable}{\ifnum \c@chapter>\z@ \thechapter-\fi \@arabic\c@table}
\makeatother

%%%% 页面设置 %%%%%%%%%%%%%%%%%%%%%%
\usepackage[bindingoffset=.5cm,centering,includeheadfoot,margin=2.5cm,headsep=1em]{geometry}
\setlength\parskip{0pt}
\usepackage{fancyhdr}\pagestyle{fancy}
\renewcommand\chaptermark[1]{%
	\markboth{\CTEXthechapter\quad #1}{}}
\renewcommand\sectionmark[1]{%
	\markright{\CTEXthesection\quad #1}}
\fancyhf{}
\fancyhead[LE,RO]{\thepage}
\fancyhead[RE]{\textsl{\nouppercase{\leftmark}}}
\fancyhead[LO]{\textsl{\nouppercase{\rightmark}}}
%\renewcommand{\headrulewidth}{0pt}
	
	
%%%% 字体 %%%%%%%%%
% xelatex 默认字体为Latin Modern
%%%%% 自定义命令 %%%%%%%%%%%%
\newcommand{\nn}{\mathbf{N}^\ast}
\renewcommand{\leq}{\leqslant}
\renewcommand{\geq}{\geqslant}
\renewcommand{\ref}{\autoref}
\renewcommand{\epsilon}{\varepsilon}
%%%%%%%%%%%%%%%%%%%%%%
%%%%%%%%%%%%%%%%%%%%%%
\begin{document}
\bibliographystyle{plainnat}
%%%%%%%%% 封面 %%%%%%%%%%%%
%\includepdf{cover/cover.pdf}
%%%%%%% 中英文摘要 %%%%%%%%%%%%%%%%
\pagenumbering{roman}
% acknowledgement

\clearpage
\thispagestyle{plain}
\phantomsection
\addcontentsline{toc}{chapter}{致谢}

\centerline{\zihao{3}\bfseries 致谢}

\linespread{1.4}\zihao{-4}
\bigskip

致谢致谢致谢致谢致谢致谢致谢致谢致谢致谢致谢致谢致谢致谢致谢致谢致谢致谢致谢致谢致谢致谢致谢致谢致谢致谢致谢致谢致谢致谢致谢致谢致谢致谢致谢致谢致谢致谢致谢致谢致谢致谢致谢致谢致谢
%%% 中文摘要
\clearpage
\thispagestyle{plain}\pagenumbering{roman}
\phantomsection
\addcontentsline{toc}{chapter}{摘\quad 要}

\centerline{\zihao{-3}\heiti 摘\quad 要}

\linespread{1.4}\zihao{-4} \bigskip

本文给出一个简易的南师大本科毕业论文\LaTeX 排版示例。

没有版权。

\bigskip

\noindent{\zihao{4}\heiti 关键词:}
\LaTeX, 排版

% Abstract
\clearpage
\thispagestyle{plain}
\phantomsection
\addcontentsline{toc}{chapter}{Abstract}

\centerline{\zihao{3}\bfseries Abstract}

\linespread{1.4}\zihao{-4}
\bigskip

This article provides a simple showcase of tysetting bachelor thesis of Nanjing Normal University with \LaTeX.

No rights reserved.

\bigskip
\noindent\textbf{\zihao{4} Keywords:} 
\LaTeX, Typesetting


\tableofcontents\newpage\mbox{}\thispagestyle{empty}\newpage
\clearpage\pagenumbering{arabic}
\include{docs/chap1}
\include{docs/chap2}
\include{docs/chap3}
\include{docs/refs}

\end{document}