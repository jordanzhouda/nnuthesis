% 使用 xelatex 编译
\documentclass[UTF8,a4paper,twoside,zihao=-4]{ctexrep}
%%%% 文档类设置 %%%%%%%%
\ctexset{
	contentsname = {目\quad 录},
	chapter = {
		beforeskip = 0pt,
		afterskip = 20pt,
		name = {第,章},
		nameformat = \zihao{3}\bfseries,
		number = \Chinese{chapter},
		numberformat = \zihao{3}\bfseries,
		titleformat = \zihao{3}\bfseries,
		fixskip = true,
		%afterindent = false
		        },
	section = {
		beforeskip = \ccwd,
		afterskip =1ex plus 0.2ex,
		format = \raggedright,
		nameformat = \zihao{4}\bfseries,
		numberformat = \zihao{4}\bfseries,
		titleformat = \zihao{4}\bfseries,
		%afterindent = false
			},
	subsection = {
		beforeskip =\ccwd,
		afterskip = 0.5ex,
		format = \raggedright,
		nameformat = \zihao{-4}\bfseries,
		numberformat = \zihao{-4}\bfseries,
		titleformat = \zihao{-4}\bfseries
			}
	    }

%%%% 宏包 %%%%%%%%%%%
\usepackage{amsmath,amssymb,amsfonts,mathrsfs}
\usepackage{graphicx}
\usepackage{tikz}
\usepackage{ulem}
\usepackage{pdfpages}
\usepackage{lipsum}
\usepackage{booktabs}
\usepackage{enumerate }
\usepackage[colorlinks,linkcolor=blue,citecolor=blue,bookmarks=true,bookmarksnumbered=true]{hyperref}
\usepackage[round]{natbib}
%%%%% 定理环境 %%%%%%%%%
\usepackage[thmmarks]{ntheorem}
{
  \theoremstyle{nonumberplain}
  \theoremheaderfont{\indent\bfseries}
  \theorembodyfont{\normalfont}
  \theoremsymbol{\ensuremath{\Box}}
  \newtheorem{proof}{证明}
}
{
  \theoremheaderfont{\indent\bfseries}
  \theorembodyfont{\normalfont}
  \newtheorem{theorem}{定理}[chapter]
  \newtheorem{lemma}{引理}[chapter]
  \newtheorem{definition}{定义}[chapter]
}
%%%%%% 图表标题设置 %%%%%%%%%%%%%%%
\usepackage{caption}
\DeclareCaptionLabelFormat{mylabel}{{\xeCJKsetup{CJKecglue={\hskip 0pt}}#1#2}}
\captionsetup{
		font= small,
		labelfont = bf,
		labelsep = space,
		labelformat = mylabel
			}
\makeatletter
\renewcommand{\thefigure}{\ifnum \c@chapter>\z@ \thechapter-\fi \@arabic\c@figure}
\renewcommand{\thetable}{\ifnum \c@chapter>\z@ \thechapter-\fi \@arabic\c@table}
\makeatother

%%%% 页面设置 %%%%%%%%%%%%%%%%%%%%%%
\usepackage[bindingoffset=.5cm,centering,includeheadfoot,margin=2.5cm,headsep=1em]{geometry}
\setlength\parskip{0pt}
\usepackage{fancyhdr}\pagestyle{fancy}
\renewcommand\chaptermark[1]{%
	\markboth{\CTEXthechapter\quad #1}{}}
\renewcommand\sectionmark[1]{%
	\markright{\CTEXthesection\quad #1}}
\fancyhf{}
\fancyhead[LE,RO]{\thepage}
\fancyhead[RE]{\textsl{\nouppercase{\leftmark}}}
\fancyhead[LO]{\textsl{\nouppercase{\rightmark}}}
%\renewcommand{\headrulewidth}{0pt}
	
	
%%%% 字体 %%%%%%%%%
% xelatex 默认字体为Latin Modern
%%%%% 自定义命令 %%%%%%%%%%%%
\newcommand{\nn}{\mathbf{N}^\ast}
\renewcommand{\leq}{\leqslant}
\renewcommand{\geq}{\geqslant}
\renewcommand{\ref}{\autoref}
\renewcommand{\epsilon}{\varepsilon}
%%%%%%%%%%%%%%%%%%%%%%
%%%%%%%%%%%%%%%%%%%%%%
\begin{document}
\bibliographystyle{plainnat}
\includepdf{cover/cover.pdf} %%%%% 封面 %%%%%%%%%%%%
%%%%%%% 中英文摘要 %%%%%%%%%%%%%%%%
\pagenumbering{roman}
%%% 中文摘要
\clearpage
\thispagestyle{plain}\pagenumbering{roman}
\phantomsection
\addcontentsline{toc}{chapter}{摘\quad 要}

\centerline{\zihao{-3}\heiti 摘\quad 要}

\linespread{1.4}\zihao{-4} \bigskip

本文给出一个简易的南师大本科毕业论文\LaTeX 排版示例。

没有版权。

\bigskip

\noindent{\zihao{4}\heiti 关键词:}
\LaTeX, 排版

% Abstract
\clearpage
\thispagestyle{plain}
\phantomsection
\addcontentsline{toc}{chapter}{Abstract}

\centerline{\zihao{3}\bfseries Abstract}

\linespread{1.4}\zihao{-4}
\bigskip

This article provides a simple showcase of tysetting bachelor thesis of Nanjing Normal University with \LaTeX.

No rights reserved.

\bigskip
\noindent\textbf{\zihao{4} Keywords:} 
\LaTeX, Typesetting


\tableofcontents\newpage\mbox{}\thispagestyle{empty}\newpage
\clearpage\pagenumbering{arabic}
\chapter{绪论}
南京师范大学坐落在六朝古都南京,是国家“211工程”重点建设的江苏省属重点大学。它的主源可追溯到1902年创办的三江师范学堂,该学堂是中国高等师范教育的发祥地之一。后历经两江优级师范学堂、南京高等师范学校、东南大学、第四中山大学、江苏大学、中央大学、南京大学等时期;其另一源头为1888年创办的汇文书院,后发展为私立金陵大学,1951年与私立金陵女子文理学院(曾称私立金陵女子大学)合并,成立公立金陵大学。1952年全国高校院系调整,在原南京大学、金陵大学等有关院系的基础上组建南京师范学院,校址设在原金陵女子大学校址。1984年改办成南京师范大学。1996年进入国家``211工程''高校行列。2000年南京动力高等专科学校并入。目前,学校正着力建设“综合性强,办学特色鲜明,国内一流的教学研究型大学”,并为今后建成“有国际影响的高水平大学”奠定坚实基础。
\begin{figure}[htb]
    \centering
    \includegraphics[scale=.6]{cover/nnu_caligraphy.png}
    \caption{南师大校名书法}\label{fig:caligraphy}
\end{figure}

南京师范大学作为一所百年老校,名家大师辈出,文化底蕴深厚。李瑞清、江谦、郭秉文、李叔同、张士一、陶行知、陈鹤琴、吴贻芳、孟宪承、徐悲鸿、高觉敷、潘玉良、张大千、唐圭璋、傅抱石、陈邦杰、陈洪、吴作人、李旭旦、孙望等诸多蜚声海内外的专家学者曾在此主政或执教。目前更有一大批国内外知名的专家学者在此潜心耕耘,著书立说,培育后学。经过一代又一代南师人薪火相继、身教言传,历史性地生成了``严谨朴实''的学术品格,育就了“以人为本”的厚生传统,砥砺出``团结奋进''的拼搏意识,塑造了``追求卓越''的创新精神。学校以``正德厚生、笃学敏行''为校训,形成了``严谨、朴实、奋发、奉献''的优良校风。
   
\section{南京师范大学简介} 
南京师范大学拥有仙林、随园、紫金三个校区,随园校区有着“东方最美丽的校园”之美誉。学校占地面积2009906平方米,建筑总面积1053697平方米。设有二级学院26个、独立学院2个。共有在职教职工3213人,专任教师1898人,其中正高级职称555人,副高级职称680人;中国科学院院士1名,国家级有突出贡献专家9名,``百千万人才工程''国家级人选9名,教育部创新团队1个、长江学者特聘教授5名,国家``千人计划''人才3名,国家``千人计划''青年人才2名,国家杰出青年科学基金获得者8名,国家级教学团队4个、国家教学名师3人,国家 “万人计划”人选6人,教育部``新世纪优秀人才支持计划''人选13人,中科院``百人计划''人选3人。共有在校普通本科生16763人,其中师范生3951人。在校研究生共10830人(学术型6212人,专业型4618人),其中博士研究生1246人,硕士研究生9584人。成人高等学历教育在籍生5216人。图书馆为全国古籍重点保护单位,总建筑面积44605平方米,馆藏纸本文献总量349.24万册,电子数据库106个。校园里拥有2 个“全国重点文物保护单位”。
\begin{figure}[htb]
    \centering
    \includegraphics[scale=.4]{cover/Nanjing_Normal_University_logo.png}
    \caption{南师大校徽}\label{fig:logo}
\end{figure}

南京师范大学充分发挥``211工程''建设的主导作用和学科学位点建设的龙头作用。目前拥有国家重点学科6个、国家重点(培育)学科3个,江苏高校优势学科10个,江苏省一级学科国家重点学科培育建设点5个,江苏省一级学科重点学科23个。2011年成立研究生院。拥有博士学位授权一级学科23个、博士学位授权二级学科专业(不含一级学科覆盖)3个,硕士学位授权一级学科37个、硕士学位授权二级学科专业(不含一级学科覆盖)10个,博士专业学位类别1个,硕士专业学位类别18个,本科招生专业(含专业类)77个,博士后科研流动站22个。学科已涉及哲、经、法、教、文、史、理、工、农、医、管、艺等门类。7个学科在全国第三轮学科评估中进入全国前十,5个学科跻身ESI全球前1\%\footnote{一个脚注}。

南京师范大学不断推进``厚生育才''战略,深化教育教学改革,提高人才培养质量。拥有国家精品课程13门、国家级精品视频公开课8门、国家级精品资源共享课15门、教育部双语教学示范课程7门、教育部来华留学英语授课品牌课程2门、全国高校职业发展与就业指导示范课程1门,教育部“马工程”重点教材相应课程“精彩一课”11门,国家特色专业8个,``十二五''本科国家级规划教材21本(部),国家级教学成果奖17项,国家级人才培养模式创新实验区3个,教育部专业综合改革试点项目1个,教育部卓越教师培养计划改革项目2个,国家实验教学示范中心2个、国家文科基础学科人才培养和科学研究基地、国家理科基础学科研究和教学人才培养基地、国家体育与艺术师资培养培训基地、教育部高校辅导员培训和研修基地、大学生文化素质教育基地、国家级虚拟仿真实验教学中心、国家大学生校外实践教育基地、国家卓越法律人才教育培养基地各1个。本科教学工作水平被教育部评为优秀,被列为江苏省内本科自主招生试点单位;学生在“挑战杯”等全国竞赛中多次获得特等奖或一等奖,4篇论文入选全国优秀博士学位论文。

南京师范大学积极实施``顶天立地''战略,科研成果追求原创,力攀高峰。拥有国家地方联合工程研究中心、教育部人文社会科学重点研究基地、教育部重点实验室、公安部重点实验室、国家体育总局体育社会科学研究基地和体育文化研究基地、全国妇联妇女与性别研究与培训基地、江苏省国家重点实验室培育建设点各1个。近些年来,获得国家社会科学基金重大项目15项、教育部哲学社会科学研究重大课题攻关项目4项、国家科技支撑计划项目2项、国家重大科学研究计划项目4项、863计划主题项目课题1项、国家自然科学基金重点项目14项;在国际权威学术期刊《科学》和《自然》上发表第一作者单位论文6篇,获国家自然科学奖二等奖1项(第一单位),教育部高等学校科学研究优秀成果奖人文社会科学一等奖6项、自然科学一等奖3项(第一单位),10部专著入选“国家社科基金优秀成果文库”,科研成果入选2008年度“中国高等学校十大科技进展”和“中国基础研究十大新闻”。我校教师主持、历时8年修订的点校本《史记》,入选《光明日报》“2013十大文化亮点”和《中华读书报》“2013十大文化事件”。

\subsection{研究与合作}
南京师范大学坚持协同创新,主动为经济建设和社会发展作贡献。拥有江苏高校协同创新中心4个,江苏省哲学社会科学重点研究基地3个,江苏新型高端智库1个,江苏省决策咨询研究基地3个,江苏省委宣传部省级重点研究基地2个,江苏高校哲学社会科学重点研究基地6个(含培育点1个)、江苏省非物质文化遗产研究基地1个、江苏省学生体质健康促进研究中心1个、江苏省老年学研究基地1个;江苏省国家重点实验室建设培育点1个、江苏省重点实验室11个、工程研究中心6个,江苏省工程实验室6个,江苏省渔业重点实验室2个,江苏省科技公共技术服务平台2个。依托优势学科和重点研究机构,并通过在地方建设一批卓有成效的产学研合作平台,促进了科技成果转化和文化创意产业发展,形成了一批富有自身特色的产学研合作领域。鼓励教师开展应用对策研究,积极发挥“智囊团”和“思想库”作用。
\begin{table}[htb]\caption{某校学生身高体重样本}\label{tb:hw}
\centering\zihao{5}
\begin{tabular}{ccccc}
	\toprule
	序号 & 性别 & 年龄 & 身高/cm & 体重 \\
	\midrule
	1 & F & 14 & 156 &42 \\
	2 & F & 16 & 158 & 45 \\
	3 & M & 14 & 162 & 48 \\
	4 & M & 15 & 163 & 50 \\
	\bottomrule 
\end{tabular}
\end{table}

南京师范大学一贯重视与海外的交流与合作,坚持国际化发展战略。学校是改革开放以后全国首批对外开放大学,是国家设立的来华留学示范基地、对外汉语教学基地、首批华文教育基地和港澳台地区幼儿教育培训基地;设有联合国教科文组织国际农村教育研究与培训中心南京基地、法国文化研究中心及南京法语培训中心、意大利文化研究中心等国际性研究和教学组织。在美国北卡罗来那州立大学、佩斯大学和法国阿尔萨斯大区建有3所孔子学院。与13所海外大学举办中外合作办学项目,学生海外学习计划学校49所。与世界上33个国家和地区的192所大学建立了校际交流关系,聘请外国专家400余人,其中长期专家56人。有来自133个国家和地区的留学生1600余人\dots

\chapter{公式排版}
微积分已有三百多年的历史,经过跨越好几个世纪的数学巨匠们的精雕细琢,千锤百炼,已经形成了一个完整的、精密的庞大知识宝库\citep{CS2016}.
\section{单调有界定理}
\begin{definition}
如果数列满足
\[
a_n\leq a_{n+1}\quad(n=1,2,\dots),
\]
则称此数列为\textbf{递增数列};如果$\{a_n\}$满足
\[
a_n\geq a_{n+1}\quad(n=1,2,\dots),
\]
则称此数列为\textbf{递减数列}.如果上面两个不等式都是严格的,即$a_n<a_{n+1}(\mbox{或}a_n>a_{n+1})(n=1,2,\dots)$,则称此数列为\textbf{严格递增}的(或严格递减的).
\end{definition}
\begin{theorem}\label{T:DDYJ}
单调且有界的数列一定有极限
\end{theorem}
\begin{proof}
	不妨设数列$\{a_n\}$是递增的且有上界.我们把这个数列的各项表示成十进制无尽小数:
\[
\begin{aligned}
	a_1&=A_1.\,b_{11}b_{12}b_{13}\dots ,\\
	a_2&=A_2.\,b_{21}b_{22}b_{23}\dots ,\\
	a_3&=A_3.\,b_{31}b_{32}b_{33}\dots ,\\
	&\dots ,
\end{aligned}
\]
其中$A_1, A_2, A_3,\dots$是整数, 而$b_{ij}(i,j=1,2,\dots)$是从0到9中的数码. 现在从上到下考察由整数$A_1,A_2,A_3,\dots$组成的那一列. 因为数列$\{a_n\}$是有界的. 这些整数不能无限地增大. 又因为这些数列是递增的,所以整数数列$\{A_n\}$在达到最大值之后将保持不变, 记这个最大的整数为$A$, 并设它在$N_0$行上出现. 现在从上往下考察第二列$b_{11},b_{21},b_{31},\dots$,不过只需要把注意力集中在第$N_0$行和以下的各行上.如果$x_1$是第$N_0$行后出现在这一列上的最大数码. 我们设它出现在第$N_1$行上, 其中$N_1\geq N_0$.那么$x_1$一旦出现将再也不会改变,这是因为$\{a_n\}$是递增数列.接着我们考察第三数列的数码$b_{12},b_{22},b_{32},\dots$.同样的讨论表明,第三列上的数码将在第$N_2\geq N_1$行及以后的各行上取一个定值$x_2$.如果我们对第四列、第五列……重复这一过程,就会得到数码$x_3,x_4,\dots$和相应的整数$N_2\leq N_3\leq\dots$.容易看出,数
\[
a=A.\,x_1x_2x_3x_4\dots
\]
应该是数列$\{a_n\}$的极限.为了证明这一结论,对任意给定的$\epsilon>0$,取$m\in\nn$,使得$10^{-m}<\epsilon$,那么对所有的$n>N_m$,$a_n$的整数部分以及小数点后的前$m$位上的数码与$a$的是一样的,因此我们有$|a_n-a|\leq10^{-m}<\epsilon$.这样就用$\epsilon-N$语言证明了
\[
\lim_{n\to\infty}a_n=A.\,x_1x_2x_3\dots
\]
\end{proof}
\section{公式交叉引用}

\section{算法环境}

\chapter{Typing English}
Just some nonsense\dots
\section{Beautiful Fonts}
\lipsum

\include{docs/refs}

\end{document}